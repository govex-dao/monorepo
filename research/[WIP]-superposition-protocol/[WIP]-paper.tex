\documentclass{article}
\usepackage{graphicx} % Required for inserting images
\usepackage{authblk}
\usepackage{hyperref}
\usepackage{algorithm}
\usepackage{algpseudocode}
\usepackage{amsmath}


% Language setting
% Replace `english' with e.g. `spanish' to change the document language
\usepackage[english]{babel}

% Set page size and margins
% Replace `letterpaper' with `a4paper' for UK/EU standard size
\usepackage[letterpaper,top=2cm,bottom=2cm,left=3cm,right=3cm,marginparwidth=1.75cm]{geometry}

% Useful packages
\usepackage{amsmath}
\usepackage{graphicx}
\title{Superposition Futarchy: Conditional Execution via Market-Based State Collapse}
\author[1]{Greshams Code}
\affil[1]{Founder, \href{https://govex.ai}{Govex.ai}}
\date{July 2025}
\begin{document}
\maketitle

\begin{abstract}
We present a novel form of futarchy where conditional outcomes exist in superposition until resolved by market consensus. Upon initiation of a futarchy proposal all proposed token actions are minted and performed immediately with conditional tokens. These tokens trade freely until being resolved by highest reading the Time-Weighted-Average-Price.
The protocol operates on a state budget rather than limiting concurrent events, allowing flexible combinations (e.g., the Cartesian product of 4 binary events or one 16-outcome event within a 16-state budget).
\end{abstract}


\section{Immediate Conditional Token Creation}
\begin{itemize}
    \item Current implementations of futarchy allow users to create proposals for a company treasury to transfer spot tokens to an address when if the proposal passes [1]. Assuming the proposal measuring period is X seconds. This creates X seconds latency between decision proposal and decision actions. This has an opportunity cost of : \begin{equation}
    C = V \cdot X \cdot r
    \end{equation}
    
    where:
    \begin{align*}
    C &= \text{Opportunity Cost} \\
    V &= \text{Value to Transfer} \\
    X &= \text{Latency (in seconds)} \\
    r &= \text{Market Interest Rate}
    \end{align*}
    Immediate transfers offer immediate capital utility.
    \item In Superposition futarchy if Alice creates a proposal for company B to pay her 1000 USDC to do work. She immediately get sent 1000 Accept-USDC. This will only be redeemable for 1000 spot USDC if the proposal passes. So she both does and doesn't get paid. The decision markets collapse the superposition to one of the options.
    \item This is a useful abstraction that allows a futarchy treasury to atomically buy back or dilute its own stock when it is trades below or above net asset value, without being front run. MntCapital an onchain fund had significant friction with buy backs [2]. This requires deep conditional liquidity, which a futarchy AMM provides [3].
   \item Conditional tokens trade freely until resolution. This helps the market to more fairly price and token actions. 

\end{itemize}

\section{Superposition states budget}
\begin{itemize}
    \item Current leading implementations of futarchy allow for N outcomes markets [1]. Other proposal will not share the same liquidity.
    \item Using carteasian product of all verses and their partitions
    \item State expansion as proposal are added to state space.
    \item Superposition: multiple states exist simultaneously
    \item States collapse as proposals resolve
    \item New proposals enter freed slots
    \item Continuous pipeline of decisions
    \item Liquidity preservation across transitions
    \item Pruning impossible combinations
    \item Lazy evaluation of state transitions
\end{itemize}

\section{References}
\begin{enumerate}
   \item https://www.govex.ai/
   \item https://metadao.fi/mtncapital/trade-v4/CV5gPgHMyJQV3a9m5FZnAxvRXsAn65dMScsANTCydHrX
   \item https://x.com/metaproph3t/status/1930686351680409637
\end{enumerate}

\end{document}